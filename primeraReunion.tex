% Created 2023-10-20 Fri 18:05
% Intended LaTeX compiler: pdflatex
\documentclass[11pt]{article}
\usepackage[utf8]{inputenc}
\usepackage[T1]{fontenc}
\usepackage{graphicx}
\usepackage{longtable}
\usepackage{wrapfig}
\usepackage{rotating}
\usepackage[normalem]{ulem}
\usepackage{amsmath}
\usepackage{amssymb}
\usepackage{capt-of}
\usepackage{hyperref}
\date{\today}
\title{}
\hypersetup{
 pdfauthor={},
 pdftitle={},
 pdfkeywords={},
 pdfsubject={},
 pdfcreator={Emacs 29.1 (Org mode 9.6.6)}, 
 pdflang={English}}
\begin{document}

\tableofcontents

\section{Primera reunion}
\label{sec:org3a75765}
Fecha: \textit{<2023-10-20 Fri>}

\subsection{Apuntes natalia}
\label{sec:org4674de3}
\begin{itemize}
\item \url{https://www.iforce2d.net/b2dtut} (Box2d)
\item \url{https://lazyfoo.net/tutorials/SDL/} (SDL2)
\item \url{https://doc.qt.io/qt-5/qtexamplesandtutorials.html} (Qt5)
\item \url{https://gameprogrammingpatterns.com/game-loop.html} (GameLoop)
\item \url{https://book-of-gehn.github.io/articles/2019/10/23/Constant-Rate-Loop.html} (Constant Rate Loop)
\item \url{https://github.com/catchorg/Catch2} (Libreria para tests automatizados)
\item \url{https://book-of-gehn.github.io/articles/2019/10/23/Constant-Rate-Loop.html} (Constant Rate Loop)
\end{itemize}
\subsection{Estructura natalia}
\label{sec:org6bcd2d5}
\begin{itemize}
\item Monitor de partidas
\item Clientes
\begin{itemize}
\item Recibe
\begin{itemize}
\item Referencia al monitor de partidas:
El que llega le dice si quiere ir a una nueva o una partida existente. Es solo al inicio. \uline{Hacer null despues de usarlo?}
\end{itemize}
\end{itemize}
\item Partida:
\begin{itemize}
\item Cada partida corre en un hilo propio
\end{itemize}
\end{itemize}

\subsubsection{Cliente lara}
\label{sec:org60a060f}
\begin{itemize}
\item Ahora la onda es \textbf{NO} bloqueante. Si no hay interaccion por x cantidad de tiempo, se cierra.
\item Poner un delay para que no haya un busy wait.
\item Recibir y enviar estan en threads distintos.
\item Como manejar los eventos?
\begin{itemize}
\item 4to hilo? Que haga de "despacher" de acciones. Cola bloqueante que le manda al render aka main. \textbf{\textbf{USAR FUNCION DE SDL}}. Gracias Juampi!!!!. Filtrar los eventos. Solo enviar al servidor los eventos que el servidor tiene que saber. Eventos como agrandar la pantalla, mutear, etc son propios del cliente.
\item Cola "no bloqueante" en el main
\end{itemize}
\item Recibe \textbf{todo} el estado del juego. DTO?
\item Acciones. Muchas que heredan de clase.
\end{itemize}

\subsection{Division de tareas recomendacion}
\label{sec:org91fd679}
\begin{enumerate}
\item Juego parte servidor. Box2d. Que se pueda mover: \textbf{Fabri}
\item Cliente.Renderizar sdl2: \textbf{Juampi}
\item Protocolo, concurrencia: \textbf{Anto}
\begin{enumerate}
\item Hay que hacer tests para serializacion y desearilazacion. Arrancar. Separar serializacion y desearilazacion de la parte de enviar.
\end{enumerate}
\end{enumerate}

\subsection{Protocolo}
\label{sec:org46b2963}
\subsubsection{Protocolo cliente->servidor}
\label{sec:org2ad4ea3}
\begin{itemize}
\item Crear partida
\item Unirse (ID)
\item Moverse (Direccion)
\end{itemize}

\subsubsection{Protocolo servidor->cliente}
\label{sec:orga0d5140}
\begin{itemize}
\item Partidas disponibles (N IDS, [ID1\ldots{}])
\item Estado del juego inicial (Con vigas + agua [splash splash])
\item Actualizacion (N Gusanos, [posx 1, posy 1, vida 1\ldots{}])
\end{itemize}
\end{document}