% Created 2023-10-21 Sat 21:23
% Intended LaTeX compiler: pdflatex
\documentclass[11pt]{article}
\usepackage[utf8]{inputenc}
\usepackage[T1]{fontenc}
\usepackage{graphicx}
\usepackage{longtable}
\usepackage{wrapfig}
\usepackage{rotating}
\usepackage[normalem]{ulem}
\usepackage{amsmath}
\usepackage{amssymb}
\usepackage{capt-of}
\usepackage{hyperref}
\date{\today}
\title{}
\hypersetup{
 pdfauthor={},
 pdftitle={},
 pdfkeywords={},
 pdfsubject={},
 pdfcreator={Emacs 29.1 (Org mode 9.6.6)}, 
 pdflang={English}}
\begin{document}

\tableofcontents

\section{Git}
\label{sec:org6dc9dbb}
\begin{itemize}
\item <>: Representa una "variable". No hay que ponerlas.
\item La gran mayoria de las opciones de un mismo comando se pueden combinar!
\end{itemize}

\subsection{Checkout}
\label{sec:orgcfe4c1b}
\subsubsection{Ir a una rama}
\label{sec:org9880c70}
\begin{verbatim}
git checkout <branch>
\end{verbatim}

\subsubsection{Ir a un commit en particular}
\label{sec:org0d25083}
\begin{verbatim}
git checkout <hash>
\end{verbatim}

\subsection{Switch}
\label{sec:org4f94128}
\subsubsection{Ir a una rama}
\label{sec:orge17a11f}
\begin{verbatim}
git switch <branch>
\end{verbatim}
\subsubsection{Crear una rama nueva}
\label{sec:org688b20b}
\begin{verbatim}
git switch -c <branch>
\end{verbatim}

\subsection{Status}
\label{sec:orgc0d2f26}
\subsubsection{Ver el estado general}
\label{sec:org2d1ce86}
\begin{verbatim}
git status
\end{verbatim}

\subsection{Log}
\label{sec:org48a8df4}
\subsubsection{Ver el historial de commits}
\label{sec:orga65e43f}
\begin{verbatim}
git log
\end{verbatim}

\subsubsection{Ver el ultimo commit}
\label{sec:org61ce05f}
\begin{verbatim}
git log -1
\end{verbatim}

\subsubsection{Ver el grafo de commits}
\label{sec:org707b543}
Para recordarlo esta buena la regla mnemotécnica "A DOG". 
\begin{verbatim}
git log --all --decorate --oneline --graph
\end{verbatim}

\subsubsection{Ver los cambios del commit}
\label{sec:org5f29363}
\begin{verbatim}
git log --patch
\end{verbatim}

\subsubsection{Grepear los mensajes de commits}
\label{sec:org92278bb}
\begin{verbatim}
git log --grep=<texto>
\end{verbatim}

\subsubsection{Grepear los diffs de commits}
\label{sec:orga1b6122}
\begin{verbatim}
git log -G <texto> --patch
\end{verbatim}

\subsubsection{Diferencias de commits entre dos ramas}
\label{sec:orga40611a}
\begin{verbatim}
git log <branchX>..<branchY>
\end{verbatim}

\subsection{Diff}
\label{sec:org0524783}
\subsubsection{Ver diferencias entre ultimo commit y el estado actual}
\label{sec:org6950fed}
\begin{verbatim}
git diff
\end{verbatim}

\subsubsection{Ver diferencias entre ultimo commit y el estado actual de un archivo particular}
\label{sec:orgd8abda5}
\begin{verbatim}
git diff <archivo>
\end{verbatim}

\subsubsection{Ver diferencias de patches entre dos ramas}
\label{sec:orgb52b3e0}
\begin{verbatim}
git diff <branchX>..<branchY>
\end{verbatim}

\subsubsection{Ver diferencias de patches entre dos ramas en un archivo especifico}
\label{sec:orgfe8ec16}
\begin{verbatim}
git diff <branchX>..<branchY> -- <archivo>
\end{verbatim}

\subsection{Show}
\label{sec:org36ed0ee}
\subsubsection{Mostrar el commit de un hash en particular}
\label{sec:org73d7401}
\begin{verbatim}
git show <hash>
\end{verbatim}

\subsection{Bisect}
\label{sec:orga6204ca}
Git de alto nivel!! Realiza una busqueda binaria interactivamente, preguntandote en cada paso. Te permite encontrar el commit "que lo rompe" de una manera re piola.

\subsubsection{Arranca el modo bisect}
\label{sec:orgaa1da55}
\begin{verbatim}
git bisect start
\end{verbatim}

\subsubsection{Marcar el primer commit bueno}
\label{sec:org0a6aea5}
\begin{verbatim}
git bisect good <hash>
\end{verbatim}

\subsubsection{Marcar el commit en el que estas como bueno}
\label{sec:org505c30c}
\begin{verbatim}
git bisect good
\end{verbatim}

\subsubsection{Marcar el commit en el que estas como malo}
\label{sec:org67a8a3c}
\begin{verbatim}
git bisect malo
\end{verbatim}

\subsubsection{Finalizar el modo bisect}
\label{sec:org7450a1c}
\begin{verbatim}
git bisect reset
\end{verbatim}

\subsection{Blame}
\label{sec:orgf9bc21f}
\subsubsection{Ver linea a linea quien es el autor del ultimo commit}
\label{sec:orgb8b542f}
\begin{verbatim}
git blame <archivo>
\end{verbatim}

\subsection{Stash}
\label{sec:org5cff059}
Este esta re bueno. Si tenes cosas sin commitear y tenes que cambiar de rama o lo que sea esto te sirve para "guardar" tus cosas momentaneamente
\subsubsection{Guardar el estado de git sin necesidad de commitear}
\label{sec:org3db7bd8}
\begin{verbatim}
git stash
\end{verbatim}

\subsubsection{Restaurar lo guardado}
\label{sec:org5fa1590}
\begin{verbatim}
git stash pop
\end{verbatim}
\end{document}